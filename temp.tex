\resheading{PROJECTS \& ACADEMIC}

\ressubheading{AI \& ML}
\\


\textbf{Optimization and RL for Adaptive Traffic Signal Control}
\hfill {[Feb'25 - Apr'25]}

\textit{Sambit Sahoo, Aditya Sinha, Adheesh Trivedi}
\hfill \href{https://github.com/AdhTri001/traffic-light-optimization}
    {Source (github.com/AdhTri001/traffic-light-optimization}

\begin{itemize}
    \item Led the development of optimization-based strategy to minimize vehicle waiting
    times at traffic signals using Self-Organizing Traffic Lights (SOTL) by
    formulating it as an ILP problem.
    \item Implemented Q-Learning and Deep Q-Learning (DQL) approaches
    for adaptive traffic control, and compared them with fixed timing approach.
    \item Conducted simulations using SUMO (Simulation of Urban MObility) to validate
    results on a 4-lane intersection under varying traffic conditions.
\end{itemize}

\textbf{Skills:} SUMO, Optimization, Reinforcement Learning (Q-learning, DQN), Traffic modeling
\\


\textbf{Texture classification and face clustering for image search}\hfill {[Sept'24-Nov'24]}

\textit{Aditya Sinha, Adheesh Trivedi}
    \hfill \href{https://github.com/AdhTri001/nomadium}
    {Source (github.com/AdhTri001/nomadium)}

\begin{itemize}
    \item The project aims to address the common challenge of navigating through
    directories containing a large collection of images,
    enabling users to efficiently filter and search for their own or others' images.
    \item \textbf{Face clustering:} \textit{MTCNN} was used for face bounding box searching.
    Then these face images were then passed to \textit{InceptionResNetV1} trained on
    \textit{VGGFACE2}, thus giving a face-to-vector embedding.
    The two faces were matched by cosin similarity of embeddings.
    Batching was implementating to prevent from running out of memory.
    \item \textbf{Texture classification:} Implemented by Aditya Sinha,
    measured different texture classification methods and search images
    based on the textual information.
\end{itemize}

\textbf{Skills:} PyTorch with CUDA\\


\textbf{A chatbot based on the BoW technique | Intermediate/+2 final project}\hfill {[Oct'21]}

\hfill \href{https://github.com/AdhTri001/tensorBot}
    {Source code (github.com/AdhTri001/tensorBot)}

\begin{itemize}
    \item A bag of words based chatbot that can perform tasks like
    \textbf{to do list, define words, note taking, fetching time
    in particular city, country or timezones, etc}.
    The bot was able to maintain context of previous messages.

    \item The sequential neural network model was trained using tensorflow,
    and the dataset for the bot was made from scratch, listing out all the
    requirements that I had, from the bot.
\end{itemize}

\textbf{Skills:} Tensorflow, PyQT5


\newpage
\ressubheading{Theoretical CS}
\\


\textbf{Tool for Model and Logical embeddings Model checking | \arpits}\hfill {[Dec’24 - Present]}

\textit{Mohit Mahapatra, Adheesh Trivedi}

\begin{itemize}
    \item Followed lecture series from NPTEL to learn about the field,
    following the book \textbf{Principles of Model checking}.
    \item Working on developing very efficient programming to convert
    action labeled model generated from tools such as \href{https://cadp.inria.fr/}{CADP}
    and \href{https://www.mcrl2.org/web/index.html}{mcrl2}, to state labelled
    models compatible with \href{https://prismmodelchecker.org/}{Prism} and
    \href{https://www.stormchecker.org/}{Storm} model checkers and vice versa.
    \item As easy it is to write a model in CADP or mcrl2,
    they don't have robust support for model checking yet.
    This projects tries to remove this gap, by bridging the
    two communities.
\end{itemize}

\textbf{Skills:} Model checking, NuSMV
\\


\textbf{Scheduling exams using Graph coloring | \pptale}\hfill {[Oct’24 - Jan'25]}

\textit{Rahul K Jana, Vivek Kumar, Ayushman Saha, Adheesh Trivedi}

\begin{itemize}
    \item We realised that our college has been making exam schedules with \textasciitilde14,000
    students and course registration entries. This required large amount of manual work.
    \item We developed a solution for generating exam schedules in universities automatically,
    through graph coloring. It can generate a close to best schedule in 15 seconds, for our institute.
    \item It can assign multiple lecture halls for certain course, if one lecture hall can't
    fulfill the strength of the course. It also tries to minimize the number of cases where
    students have to give more than one exam in 24 hours.
\end{itemize}

\textbf{Skills:} Python, NetworkX, PyInstaller
\\


\textbf{Reading project on Graph theory | \pptale}\hfill {[Jan'24-Apr'24]}

\textit{Tejal R, Adheesh Trivedi}

\begin{itemize}
    \item Read the book \textbf{A First look at Graph Theory} by \textit{Clark and Halton}
    \item Solved exercises in the book and test the learned concepts of Graph Theory
\end{itemize}


\ressubheading{Other projects}
\\


\textbf{Litesoph | \vardha}\hfill {[Feb'24-Aug’24]}

\textit{Jay Bharambe, Adheesh Trivedi, Vibha Dinesh Udupa, Ranjan Bassi}
    \hfill \href{https://github.com/aitgcodes/litesoph}
    {Source (github.com/aitgcodes/litesoph)}

\begin{itemize}
    \item Utilizing engines like \href{www.octopus-code.org}{Octopus},
    \href{gpaw.readthedocs.io}{GPAW}, \href{www.nwchem-sw.org}{NWChem}
    and launching a job in clusters like \href{hpc.iitr.ac.in}{param-ganga}
    and \href{www.iitg.ac.in/hpc/home}{param-kamrupa}, some of the most powerful
    HPC in India.
    \item Layer Integrated Toolkit and Engine for Simulations of Photo-induced Phenomena, written entirely in Python.
    It integrates such engines, utilizing their ground state and DFT computation capabilities.
    \item Worked on Litesoph as a lead developer intern.
\end{itemize}

\textbf{Skills:} Tkinter, Using HPC (High Performance Computing), SLURM
\\


\textbf{A general purpose discord bot}\hfill {[Apr'21-Jun'21]}

\hfill \href{https://github.com/AdhTri001/UFO-BOT}
    {Source code (github.com/AdhTri001/UFO-BOT)}

\begin{itemize}
    \item The bot had features that included
    \textbf{ server moderation, games, music playback, jokes, and meme creations}.
    It was designed using discord.py and was written in python.
    \item The project was written in Python and utilized PostgreSQL database to store every configuration for the bot,
    which was server and user specific. The bot was designed to be scalable.
\end{itemize}


\newpage
\resheading{\textbf{HIGHLIGHTED SKILLS}}
 \begin{itemize}
\setlength\itemsep{-0.45cm}
    \item \textbf{Programming Languages}: Advanced Python, Intermediate C++, \LaTeX\\
    % \item \textbf{Tools}: Blender (for modeling \& animating), Davinci (for video editing), GNU/Linux knowledge \\
    \item \textbf{Tools}: GNU/Linux shell, Pandas, numpy, matplotlib, \\
\end{itemize}